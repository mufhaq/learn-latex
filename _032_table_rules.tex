\documentclass{article}

\usepackage{parskip}
\usepackage{array}
\usepackage{xcolor}
\usepackage{babel}

\usepackage{booktabs}
% booktabs provides four different types of lines. Each of those commands has to
% be used as the first thing in a row or following another rule. Three of the
% rule commands are: \toprule, \midrule, and \bottomrule. From their names the
% intended place of use should be clear:

\begin{document}
Table rules
% https://www.learnlatex.org/en/lesson-08#adding-rules-lines

\begin{tabular}{lll}
    \toprule
    Animal  & Food  & Size      \\
    \midrule
    dog     & meat  & medium    \\
    horse   & hay   & large     \\
    frog    & flies & small     \\
    \bottomrule
\end{tabular}

\textcolor{blue}{\textbf{Specific Column with \textbackslash cmidrule}}

% The fourth rule command provided by booktabs is \cmidrule. It can be used to
% draw a rule that doesn’t span the entire width of the table but only a
% specified column range
\begin{tabular}{*{3}{l}}
    \toprule
    Animal  & Food  & Size      \\
    \midrule
    dog     & meat  & medium    \\
    \cmidrule{1-2}
    horse   & hay   & large     \\
    \cmidrule{1-1}
    \cmidrule{3-3}
    frog    & flies & small     \\
    \bottomrule
\end{tabular}

\textcolor{blue}{\textbf{Another Feature of \textbackslash cmidrule}}

% r and l mean the rule is shortened on its right and left end, respectively.
\begin{tabular}{*{3}{l}}
    \toprule
    Animal  & Food  & Size      \\
    \midrule
    dog     & meat  & medium    \\
    \cmidrule{1-2}
    horse   & hay   & large     \\
    \cmidrule(r){1-1}
    \cmidrule(rl){2-2}
    \cmidrule(l){3-3}
    frog    & flies & small     \\
    \bottomrule
\end{tabular}

\textcolor{blue}{\textbf{Insert Small Skip Whitespace with \textbackslash addlinespace}}

\textbf{Without \textbackslash addlinespace}

\begin{tabular}{cp{9cm}}
    Animal  & Description   \\
    dog     & Member of the genus Canis, which forms part of the wolf-like 
              canids, and is the most widely abundant terrestrial carnivore. \\
    cat     & Domestic species of small carnivorous mammal. It is the only
              domesticated species in the family Felidae and is often referred
              to as the domestic cat to distinguish it from the wild members of
              the family. \\
\end{tabular}


\textbf{With \textbackslash addlinespace}

\begin{tabular}{cp{9cm}}
    Animal  & Description   \\
    dog     & Member of the genus Canis, which forms part of the wolf-like 
              canids, and is the most widely abundant terrestrial carnivore. \\
    \addlinespace
    cat     & Domestic species of small carnivorous mammal. It is the only
              domesticated species in the family Felidae and is often referred
              to as the domestic cat to distinguish it from the wild members of
              the family. \\
\end{tabular}
\end{document}
