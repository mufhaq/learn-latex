\documentclass{article}


\usepackage{parskip}
\usepackage{array} % More functionality for LaTeX tables
\usepackage{xcolor}

\begin{document}
Tables
% https://www.learnlatex.org/en/lesson-08

% Tabular environment
\begin{tabular}{lll} % For 3 left aligned column
    % In a table body columns are separated using an ampersand `&` and a new row 
    % is started using `\\`.
    Animal  & Food  & Size      \\
    dog     & meat  & medium    \\
    horse   & hay   & large     \\
    frog    & flise & small     \\
\end{tabular}

\newpage
% If a table column contains a lot of text you will have issues to get that 
% right with only l, c, and r

% The issue is that the l type column typesets its contents in a single row at 
% its natural width, even if there is a page border in the way
\textbf{\textcolor{red}{ISSUE}}

\begin{tabular}{cl}
    Animal  & Description   \\
    dog     & Member of the genus Canis, which forms part of the wolf-like 
              canids, and is the most widely abundant terrestrial carnivore. \\
    cat     & Domestic species of small carnivorous mammal. It is the only
              domesticated species in the family Felidae and is often referred
              to as the domestic cat to distinguish it from the wild members of
              the family. \\
\end{tabular}

\textbf{\textcolor{blue}{FIXED}}

% To overcome this you can use the p column
\begin{tabular}{cp{9cm}}
    Animal  & Description   \\
    dog     & Member of the genus Canis, which forms part of the wolf-like 
              canids, and is the most widely abundant terrestrial carnivore. \\
    cat     & Domestic species of small carnivorous mammal. It is the only
              domesticated species in the family Felidae and is often referred
              to as the domestic cat to distinguish it from the wild members of
              the family. \\
\end{tabular}

\newpage

% If your table has many columns of the same type it is cumbersome to put that 
% many column definitions in the preamble

% You can make things easier by using *{num}{string}, which repeats the string 
% num times

\begin{tabular}{*{3}{l}} % Equal to lll
    Animal  & Food  & Size      \\
    dog     & meat  & medium    \\
    horse   & hay   & large     \\
    frog    & flise & small     \\
\end{tabular}
\end{document}
