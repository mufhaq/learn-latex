\documentclass{article}

\usepackage{parskip}
\usepackage{array}
\usepackage{tabularx}

\begin{document}
tabularx
% https://www.learnlatex.org/en/more-08#tabularx

\section{tabular}
\begin{center}
\begin{tabular}{|l|p{2cm}|}
    \hline
    A & B B B B B B B B B B B B B B B B B B B B B B B B \\
    C & D D D D D D D                                   \\
    \hline
\end{tabular}
\end{center}

\section{tabularx with 0.5 textwidth}
% The tabularx environment, provided by the package of the same name, has a
% similar syntax to tabular* but instead of adjusting the inter-column space,
% adjusts the widths of columns specified by a new column type, X. This is
% equivalent to a specification of p{...} for an automatically determined width.
\begin{center}
\begin{tabularx}{.5\textwidth}{|l|X|}
    \hline
    A & B B B B B B B B B B B B B B B B B B B B B B B B \\
    C & D D D D D D D                                   \\
    \hline
\end{tabularx}
\end{center}

\section{tabularx with empty textwidth}
\begin{center}
\begin{tabularx}{\textwidth}{|l|X|}
    \hline
    A & B B B B B B B B B B B B B B B B B B B B B B B B \\
    C & D D D D D D D                                   \\
    \hline
\end{tabularx}
\end{center}

\section{tabularx vs tabular* vs tabular* with \\ \textbackslash extracolsep \textbackslash fill}
\begin{center}
\begin{tabularx}{\textwidth}{|l|X|}
    \hline
    A & B B B B B B B B B B B B B B B B B B B B B B B B \\
    C & D D D D D D D                                   \\
    \hline
\end{tabularx}

\begin{tabular*}{\textwidth}{|l|l|}
    \hline
    A & B B B B B B B B B B B B B B B B B B B B B B B B \\
    C & D D D D D D D                                   \\
    \hline
\end{tabular*}

\begin{tabular*}{\textwidth}{@{\extracolsep{\fill}}|l|l|}
    \hline
    A & B B B B B B B B B B B B B B B B B B B B B B B B \\
    C & D D D D D D D                                   \\
    \hline
\end{tabular*}
\end{center}
\end{document}
