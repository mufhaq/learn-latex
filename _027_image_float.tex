\documentclass[a4paper]{article}

\usepackage{graphicx}
\usepackage{parskip}
\usepackage{lipsum}

\begin{document}
Making images float
% https://www.learnlatex.org/en/lesson-07\#making-images-float

% Traditionally in typesetting, particularly with technical documents, graphics
% may move to another spot in the document. This is called a float. Images are
% normally included as floats so they do not leave large gaps in the page.


\lipsum[1-3]

Test location.
\begin{figure}[ht]
    \centering % Use this instead of center environment
    \includegraphics[width=0.5\textwidth]{LaTeX_logo.svg.png} % Relative size
    \caption{An exmplae image}
\end{figure}

% h ‘Here’ (if possible)
% t Top of the page
% b Bottom of the page
% p A dedicated page only for floats

% Inside a float, you should use \centering if you want to horizontally center
% content; this avoids both the float and center environment adding extra
% vertical space.

\lipsum[4-8]
\end{document}
